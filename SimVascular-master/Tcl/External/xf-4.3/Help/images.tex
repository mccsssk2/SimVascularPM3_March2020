\batchmode
\documentstyle[makeidx,supertab]{report}
\makeatletter

\input{epsf}




\oddsidemargin .3cm
\evensidemargin -.3cm
\topmargin -1.2cm
\textheight 22.5cm
\textwidth 16cm

\setcounter{topnumber}{5}
\setcounter{totalnumber}{5}
\setcounter{bottomnumber}{5}









\hyphenation{
Con-fig-u-ra-tion
In-ter-pret-er
Brow-ser
Objekt-browser
Objekt-browsers
Soft-ware
icon-bar
menu-bar
wid-get
scroll-bar
top-level
fore-ground
back-ground
STONE-Be-nutz-ungs-ober-flae-che
Be-nutz-ungs-ober-flae-che
pop-ped
}










































































\makeindex


\def\mymedskip{\medskip}

\def\myclearpage{\clearpage}

\def\epsftmpsize#1{0.9#1}

\def\epsfsize#1#2{
  \ifdim\epsftmpsize#1>\hsize\hsize\else\epsftmpsize#1\fi}

\newcommand {\XF}{{\em XF }}

\newcommand {\XFS}{{\em XF}}

\newcommand {\XASK}{{\em XASK }}

\newcommand {\XASKS}{{\em XASK}}

\newcommand {\BYO}{{\em BYO }}

\newcommand {\BYOS}{{\em BYO}}

\newcommand {\TCL}{{\em Tcl }}

\newcommand {\TCLS}{{\em Tcl}}

\newcommand {\TK}{{\em Tk }}

\newcommand {\TKS}{{\em Tk}}

\newcommand {\TCLTK}{{\em Tcl/Tk }}

\newcommand {\TCLTKS}{{\em Tcl/Tk}}

\newcommand {\PACKER}{{\em packer }}

\newcommand {\PACKERS}{{\em packer}}

\newcommand {\PLACER}{{\em placer }}

\newcommand {\PLACERS}{{\em placer}}

\newcommand {\ATFSLONGSHORT}{{\em Attributed File System}}

\newcommand {\ATFSLONG}{{\em Attributed File System }}

\newcommand {\ATFSSHORT}{{\em AtFS}}

\newcommand {\ATFS}{{\em AtFS }}

\newcommand {\OBJBROWSHORT}{{\em Objekt\-browser}}

\newcommand {\OBJBROWS}{{\em Objekt\-browsers }}

\newcommand {\OBJBROW}{{\em Objekt\-browser }}

\newcommand {\SDESSHORT}{{\em SPU's }}

\newcommand {\SDESHORT}{{\em SPU}}

\newcommand {\SDESLONGNORM}{Soft\-ware-Pro\-duk\-tions\-um\-ge\-bungen }

\newcommand {\SDESLONG}{{\em Soft\-ware-Pro\-duk\-tions\-um\-ge\-bungen }}

\newcommand {\SDELONG}{{\em Soft\-ware-Pro\-duk\-tions\-um\-ge\-bung }}

\newcommand {\SDES}{{\em SPU's }}

\newcommand {\SDE}{{\em SPU }}

\newcommand {\SOFTOBJSSHORT}{{\em Software\-objekts}}

\newcommand {\SOFTOBJESHORT}{{\em Software\-objekte}}

\newcommand {\SOFTOBJSHORT}{{\em Software\-objekt}}

\newcommand {\SOFTOBJENSHORT}{{\em Software\-objekten}}

\newcommand {\SOFTOBJS}{{\em Software\-objekts }}

\newcommand {\SOFTOBJEN}{{\em Software\-objekten }}

\newcommand {\SOFTOBJE}{{\em Software\-objekte }}

\newcommand {\SOFTOBJ}{{\em Software\-objekt }}

\newcommand {\STAGESHORT}{{\sc Stage}}

\newcommand {\STAGE}{{\sc Stage }}

\newcommand {\STONESHORT}{{\em Stone}}

\newcommand {\STONE}{{\em Stone }}

\newcommand {\HPES}{{\em HP Encapsulator$^{TM}$}}

\newcommand {\HPE}{{\em HP Encapsulator$^{TM}$ }}

\newcommand {\XWSS}{{\em X window system$^{TM}$}}

\newcommand {\XWS}{{\em X window sys\-tem$^{TM}$ }}

\newcommand {\UNIXS}{{\em $UNIX^{TM}$}}

\newcommand {\UNIX}{{\em $UNIX^{TM}$ }}

\newcommand {\MOTIFS}{{\em $Motif^{TM}$}}

\newcommand {\MOTIF}{{\em $Motif^{TM}$ }}

\newcommand {\DOSS}{{\em $MS-DOS^{TM}$}}

\newcommand {\DOS}{{\em $MS-DOS^{TM}$ }}

\newcommand {\BOFL}{{Be\-nut\-zungs\-ober\-fl"a\-che }}

\newcommand {\BOFLS}{{Be\-nut\-zungs\-ober\-fl"a\-chen }}

\newcommand {\BOFLSHORT}{{Be\-nut\-zungs\-ober\-fl"a\-che}}

\newcommand {\BOFLSSHORT}{{Be\-nut\-zungs\-ober\-fl"a\-chen}}

\makeatother
\newenvironment{tex2html_wrap}{}{}
\newwrite\lthtmlwrite
\def\lthtmltypeout#1{{\let\protect\string\immediate\write\lthtmlwrite{#1}}}%
\newbox\sizebox
\begin{document}
\pagestyle{empty}
\setcounter{topnumber}{5}
\setcounter{totalnumber}{5}
\setcounter{bottomnumber}{5}
\setcounter{topnumber}{5}
\setcounter{totalnumber}{5}
\setcounter{bottomnumber}{5}
{\newpage
\clearpage
\samepage \setbox\sizebox=\hbox{$Motif^{TM}$}\lthtmltypeout{latex2htmlSize :tex2html_wrap_inline2340: \the\ht\sizebox::\the\dp\sizebox.}\box\sizebox
}

\stepcounter{chapter}
{\newpage
\clearpage
\samepage \setbox\sizebox=\hbox{$^{TM}$}\lthtmltypeout{latex2htmlSize :tex2html_wrap_inline2342: \the\ht\sizebox::\the\dp\sizebox.}\box\sizebox
}

\stepcounter{section}
\stepcounter{section}
\stepcounter{section}
\stepcounter{chapter}
{\newpage
\clearpage
\samepage \setbox\sizebox=\hbox{$^{TM}$}\lthtmltypeout{latex2htmlSize :tex2html_wrap_inline2350: \the\ht\sizebox::\the\dp\sizebox.}\box\sizebox
}

\stepcounter{section}
\stepcounter{section}
{\newpage
\clearpage
\samepage \setbox\sizebox=\hbox{$Motif^{TM}$}\lthtmltypeout{latex2htmlSize :tex2html_wrap_inline2352: \the\ht\sizebox::\the\dp\sizebox.}\box\sizebox
}

{\newpage
\clearpage
\samepage \setbox\sizebox=\hbox{$Motif^{TM}$}\lthtmltypeout{latex2htmlSize :tex2html_wrap_inline2354: \the\ht\sizebox::\the\dp\sizebox.}\box\sizebox
}

{\newpage
\clearpage
\samepage \setbox\sizebox=\hbox{$Motif^{TM}$}\lthtmltypeout{latex2htmlSize :tex2html_wrap_inline2356: \the\ht\sizebox::\the\dp\sizebox.}\box\sizebox
}

\stepcounter{section}
{\newpage
\clearpage
\samepage \setbox\sizebox=\hbox{$Motif^{TM}$}\lthtmltypeout{latex2htmlSize :tex2html_wrap_inline2358: \the\ht\sizebox::\the\dp\sizebox.}\box\sizebox
}

{\newpage
\clearpage
\samepage \begin{figure}[ht]
  \centerline{
  \epsfysize=6cm
  \epsffile{pictures/design/BYO.ips}}

  \label{fig:BYO design}
\end{figure}
}

{\newpage
\clearpage
\samepage \begin{figure}[ht]
  \centerline{
  \epsfysize=6cm
  \epsffile{pictures/design/XF.ips}}

  \label{fig:XF design}
\end{figure}
}

\stepcounter{chapter}
{\newpage
\clearpage
\samepage \setbox\sizebox=\hbox{$Motif^{TM}$}\lthtmltypeout{latex2htmlSize :tex2html_wrap_inline2360: \the\ht\sizebox::\the\dp\sizebox.}\box\sizebox
}

{\newpage
\clearpage
\samepage \setbox\sizebox=\hbox{$UNIX^{TM}$}\lthtmltypeout{latex2htmlSize :tex2html_wrap_inline2362: \the\ht\sizebox::\the\dp\sizebox.}\box\sizebox
}

{\newpage
\clearpage
\samepage \setbox\sizebox=\hbox{$^{TM}$}\lthtmltypeout{latex2htmlSize :tex2html_wrap_inline2364: \the\ht\sizebox::\the\dp\sizebox.}\box\sizebox
}

\stepcounter{section}
\stepcounter{subsection}
\stepcounter{subsection}
\stepcounter{subsection}
\stepcounter{subsection}
\stepcounter{subsection}
{\newpage
\clearpage
\samepage \setbox\sizebox=\hbox{$\backslash$}\lthtmltypeout{latex2htmlSize :tex2html_wrap_inline2366: \the\ht\sizebox::\the\dp\sizebox.}\box\sizebox
}

\stepcounter{section}
{\newpage
\clearpage
\samepage \setbox\sizebox=\hbox{$Motif^{TM}$}\lthtmltypeout{latex2htmlSize :tex2html_wrap_inline2368: \the\ht\sizebox::\the\dp\sizebox.}\box\sizebox
}

\stepcounter{subsection}
\stepcounter{subsection}
{\newpage
\clearpage
\samepage \setbox\sizebox=\hbox{$UNIX^{TM}$}\lthtmltypeout{latex2htmlSize :tex2html_wrap_inline2370: \the\ht\sizebox::\the\dp\sizebox.}\box\sizebox
}

\stepcounter{subsection}
\stepcounter{subsection}
\stepcounter{subsection}
\stepcounter{subsubsection}
{\newpage
\clearpage
\samepage \begin{figure}[ht]
  \centerline{
  \epsfysize=9cm
  \epsffile{pictures/tclTk/placer1.ps}}

  \label{fig:Placed widgets}
\end{figure}
}

\stepcounter{subsubsection}
{\newpage
\clearpage
\samepage \begin{figure}[ht]
  \centerline{
  \epsfysize=6cm
  \epsffile{pictures/tclTk/packer1.ps}}

  \label{fig:Packed widgets}
\end{figure}
}

\stepcounter{subsection}
\stepcounter{chapter}
\stepcounter{section}
\stepcounter{subsection}
\stepcounter{subsection}
\stepcounter{subsection}
\stepcounter{subsubsection}
\stepcounter{subsubsection}
\stepcounter{subsection}
\stepcounter{subsection}
\stepcounter{section}
\stepcounter{subsection}
\stepcounter{subsection}
\stepcounter{subsection}
\stepcounter{subsection}
\stepcounter{subsection}
\stepcounter{subsection}
\stepcounter{section}
\stepcounter{subsection}
\stepcounter{subsection}
\stepcounter{section}
\stepcounter{subsection}
\stepcounter{subsection}
\stepcounter{subsection}
\stepcounter{subsection}
{\newpage
\clearpage
\samepage \begin{tabular}{|l|p{8.5cm}|} \hline
Level & Purpose\\  \hline
1     & Not used by {\em XF}
.\\  \hline
2     & Not used by {\em XF}
.\\  \hline
3     & Not used by {\em XF}
.\\  \hline
4     & Procedures that are used to implement the {\em XF }
 alias
         feature get this level. They should be saved, but it
         is not necessary to display them.\\  \hline
5     & The main template procedures (those which are called
         by the user) have this level.\\  \hline
6     & The supporting template procedures (those which are
         not called by the user) have this level. They should
         not be displayed.\\  \hline
7     & Procedures that are used by the {\em XF }
 generated code
         get this level. They should be saved, but it is not
         necessary to display them.\\  \hline
8     & {\em Tk }
 procedures that are to be saved get this level.
         Right now no procedure has this level.\\  \hline
9     & Procedures that have this level are totally ignored
         by {\em XF}
.\\  \hline
\end{tabular}
}

\stepcounter{chapter}
{\newpage
\clearpage
\samepage \begin{figure}[ht]
  \centerline{
  \epsfysize=6cm
  \epsffile{pictures/design/BYO.ips}}

  \label{fig:BYO design}
\end{figure}
}

{\newpage
\clearpage
\samepage \begin{figure}[ht]
  \centerline{
  \epsfysize=6cm
  \epsffile{pictures/design/XF.ips}}

  \label{fig:XF design}
\end{figure}
}

\stepcounter{section}
\stepcounter{section}
\stepcounter{subsection}
\stepcounter{subsubsection}
\stepcounter{subsubsection}
\stepcounter{subsubsection}
{\newpage
\clearpage
\samepage \begin{supertabular}{|l|p{8.5cm}|}
0      & Activates the packing dialog for this widget.\\  \hline
1      & Activates the placing dialog for this widget.\\  \hline
2      & Activates the default geometry handling dialog for
         this widget.\\  \hline
3      & Activates the binding dialog for this widget.\\  \hline
4      & Activates the default parameter dialog for this
         widget. This dialog should cover the most important
         resources of the widget.\\  \hline
5      & Activates the special parameter dialog for this
         widget. This dialog is used to implement special
         features, like drawing for canvas widgets.\\  \hline
\end{supertabular}
}

\stepcounter{subsubsection}
\stepcounter{subsubsection}
\stepcounter{subsection}
\stepcounter{subsection}
\stepcounter{chapter}
\stepcounter{section}
{\newpage
\clearpage
\samepage \setbox\sizebox=\hbox{$^{TM}$}\lthtmltypeout{latex2htmlSize :tex2html_wrap_inline2450: \the\ht\sizebox::\the\dp\sizebox.}\box\sizebox
}

{\newpage
\clearpage
\samepage \setbox\sizebox=\hbox{$^{TM}$}\lthtmltypeout{latex2htmlSize :tex2html_wrap_inline2452: \the\ht\sizebox::\the\dp\sizebox.}\box\sizebox
}

{\newpage
\clearpage
\samepage \setbox\sizebox=\hbox{$UNIX^{TM}$}\lthtmltypeout{latex2htmlSize :tex2html_wrap_inline2454: \the\ht\sizebox::\the\dp\sizebox.}\box\sizebox
}

{\newpage
\clearpage
\samepage \setbox\sizebox=\hbox{$MS-DOS^{TM}$}\lthtmltypeout{latex2htmlSize :tex2html_wrap_inline2456: \the\ht\sizebox::\the\dp\sizebox.}\box\sizebox
}

\stepcounter{section}
{\newpage
\clearpage
\samepage \setbox\sizebox=\hbox{$Motif^{TM}$}\lthtmltypeout{latex2htmlSize :tex2html_wrap_inline2458: \the\ht\sizebox::\the\dp\sizebox.}\box\sizebox
}

\stepcounter{chapter}
\stepcounter{section}
{\newpage
\clearpage
\samepage \begin{figure}[hbt]
  \centerline{
  \epsfysize=9cm
  \epsffile{pictures/external/edge.ps}}

  \label{fig:The edge program}
\end{figure}
}

\stepcounter{section}
\stepcounter{subsection}
\stepcounter{subsection}
\stepcounter{subsection}
\stepcounter{section}
\stepcounter{subsection}
\stepcounter{subsection}
\stepcounter{section}
{\newpage
\clearpage
\samepage \begin{figure}[hbt]
  \centerline{
  \epsfysize=9.5cm
  \epsffile{pictures/external/xfappdef.ps}}

  \label{fig:The xfappdef program}
\end{figure}
}

\stepcounter{section}
{\newpage
\clearpage
\samepage \begin{figure}[hbt]
  \centerline{
  \epsfysize=8.5cm
  \epsffile{pictures/external/xfhardcopy.ps}}

  \label{fig:The xfhardcopy program}
\end{figure}
}

\stepcounter{subsection}
{\newpage
\clearpage
\samepage \begin{tabular}{|l|l|p{8.5cm}|} \hline
Variable name & Contents\\  \hline
height        & The height of the selected widget\\  \hline
id            & The X window id of the selected widget\\  \hline
outputFile    & The name of the specified output file\\  \hline
rootx         & The absolute x position of the widget\\  \hline
rooty         & The absolute y position of the widget\\  \hline
widget        & The {\em Tk }
 widget name of the selected widget\\  \hline
width         & The width of the selected widget\\  \hline
x             & The relative x position of the widget\\  \hline
y             & The relative y position of the widget\\  \hline
\end{tabular}
}

\stepcounter{section}
{\newpage
\clearpage
\samepage \begin{figure}[hbt]
  \centerline{
  \epsfysize=12cm
  \epsffile{pictures/external/xfhelp.ps}}

  \label{fig:The xfhelp program}
\end{figure}
}

\stepcounter{subsection}
\stepcounter{section}
{\newpage
\clearpage
\samepage \begin{figure}[hbt]
  \centerline{
  \epsfysize=12.5cm
  \epsffile{pictures/external/xfpixmap.ps}}

  \label{fig:The xfpixmap program}
\end{figure}
}

\stepcounter{section}
{\newpage
\clearpage
\samepage \begin{figure}[hbt]
  \label{fig:The xftutorial program}
  \centerline{
  \epsfysize=11cm
  \epsffile{pictures/external/xftutorial.ps}}

\end{figure}
}

\stepcounter{subsection}
\stepcounter{chapter}
\stepcounter{section}
\stepcounter{subsection}
{\newpage
\clearpage
\samepage \begin{figure}[hbt]
  \centerline{
  \epsfysize=11cm
  \epsffile{pictures/dialog/main.ps}}

  \label{fig:The procedure XFProcMain}
\end{figure}
}

\stepcounter{section}
\stepcounter{subsection}
{\newpage
\clearpage
\samepage \begin{figure}[hbt]
  \centerline{
  \epsfysize=4cm
  \epsffile{pictures/dialog/read.ps}}

  \label{fig:The procedure XFProcFileEnterTCL}
\end{figure}
}

\stepcounter{subsection}
\stepcounter{subsection}
{\newpage
\clearpage
\samepage \begin{figure}[hbt]
  \centerline{
  \epsfysize=4cm
  \epsffile{pictures/dialog/new.ps}}

  \label{fig:The procedure XFProcFileNew}
\end{figure}
}

\stepcounter{subsection}
{\newpage
\clearpage
\samepage \begin{figure}[hbt]
  \centerline{
  \epsfysize=8cm
  \epsffile{pictures/dialog/load.ps}}

  \label{fig:The procedure XFProcFileLoad}
\end{figure}
}

\stepcounter{subsection}
{\newpage
\clearpage
\samepage \begin{figure}[hbt]
  \centerline{
  \epsfysize=4cm
  \epsffile{pictures/dialog/quit.ps}}

  \label{fig:The procedure XFProcFileQuit}
\end{figure}
}

\stepcounter{subsection}
\stepcounter{subsection}
\stepcounter{section}
\stepcounter{subsection}
\stepcounter{subsection}
{\newpage
\clearpage
\samepage \begin{figure}[hbt]
  \centerline{
  \epsfysize=10cm
  \epsffile{pictures/dialog/binding.ps}}

  \label{fig:The procedure XFProcConfBinding}
\end{figure}
}

\stepcounter{subsection}
\stepcounter{subsection}
\stepcounter{subsection}
\stepcounter{subsection}
\stepcounter{subsection}
\stepcounter{subsection}
\stepcounter{subsection}
\stepcounter{subsection}
{\newpage
\clearpage
\samepage \begin{figure}[hbt]
  \centerline{
  \epsfysize=4cm
  \epsffile{pictures/dialog/layout.ps}}

  \label{fig:The procedure XFProcConfLayout}
\end{figure}
}

\stepcounter{subsection}
{\newpage
\clearpage
\samepage \begin{figure}[hbt]
  \centerline{
  \epsfysize=10.5cm
  \epsffile{pictures/dialog/packing.ps}}

  \label{fig:The procedure XFProcConfPacking}
\end{figure}
}

\stepcounter{subsection}
\stepcounter{subsection}
{\newpage
\clearpage
\samepage \begin{figure}[hbt]
  \centerline{
  \epsfysize=9.5cm
  \epsffile{pictures/dialog/general.ps}}

  \label{fig:The procedure XFProcConfParametersGeneral}
\end{figure}
}

\stepcounter{subsection}
{\newpage
\clearpage
\samepage \begin{figure}[hbt]
  \centerline{
  \epsfysize=9cm
  \epsffile{pictures/dialog/groups.ps}}

  \label{fig:The procedure XFProcConfParametersGroups}
\end{figure}
}

\stepcounter{subsection}
{\newpage
\clearpage
\samepage \begin{figure}[hbt]
  \centerline{
  \epsfysize=9cm
  \epsffile{pictures/dialog/params.ps}}

  \label{fig:The procedure XFProcConfParametersSmall}
\end{figure}
}

\stepcounter{subsection}
\stepcounter{subsection}
{\newpage
\clearpage
\samepage \begin{figure}[hbt]
  \centerline{
  \epsfysize=12.5cm
  \epsffile{pictures/dialog/placing.ps}}

  \label{fig:The procedure XFProcConfPlacing}
\end{figure}
}

\stepcounter{section}
\stepcounter{subsection}
\stepcounter{subsection}
\stepcounter{subsection}
\stepcounter{subsection}
\stepcounter{subsection}
\stepcounter{subsection}
\stepcounter{subsection}
\stepcounter{subsection}
\stepcounter{subsection}
\stepcounter{subsection}
\stepcounter{subsection}
{\newpage
\clearpage
\samepage \begin{figure}[hbt]
  \centerline{
  \epsfysize=4.5cm
  \epsffile{pictures/dialog/cuttree.ps}}

  \label{fig:The procedure XFProcEditShowCut (tree)}
\end{figure}
}

{\newpage
\clearpage
\samepage \begin{figure}[hbt]
  \centerline{
  \epsfysize=7cm
  \epsffile{pictures/dialog/cutscript.ps}}

  \label{fig:The procedure XFProcEditShowCut (script)}
\end{figure}
}

\stepcounter{section}
\stepcounter{subsection}
{\newpage
\clearpage
\samepage \begin{figure}[hbt]
  \centerline{
  \epsfysize=12.5cm
  \epsffile{pictures/dialog/commands.ps}}

  \label{fig:The procedure XFProcProgCommands}
\end{figure}
}

{\newpage
\clearpage
\samepage \begin{figure}[hbt]
  \centerline{
  \epsfysize=3cm
  \epsffile{pictures/dialog/procSave.ps}}

  \label{fig:The procedure XFProcProgCommands (saving)}
\end{figure}
}

{\newpage
\clearpage
\samepage \begin{figure}[hbt]
  \centerline{
  \epsfysize=7.5cm
  \epsffile{pictures/dialog/procLoad.ps}}

  \label{fig:The procedure XFProcProgCommands (loading)}
\end{figure}
}

\stepcounter{subsection}
{\newpage
\clearpage
\samepage \begin{figure}[hbt]
  \centerline{
  \epsfysize=7cm
  \epsffile{pictures/dialog/editScript.ps}}

  \label{fig:The procedure XFProcProgEditScript}
\end{figure}
}

\stepcounter{subsection}
\stepcounter{subsection}
{\newpage
\clearpage
\samepage \begin{figure}[hbt]
  \centerline{
  \epsfysize=5cm
  \epsffile{pictures/dialog/errors.ps}}

  \label{fig:The procedure XFProcProgErrors}
\end{figure}
}

\stepcounter{subsection}
{\newpage
\clearpage
\samepage \begin{figure}[hbt]
  \centerline{
  \epsfysize=10cm
  \epsffile{pictures/dialog/globals.ps}}

  \label{fig:The procedure XFProcProgGlobals}
\end{figure}
}

\stepcounter{subsection}
{\newpage
\clearpage
\samepage \begin{figure}[hbt]
  \centerline{
  \epsfysize=12.5cm
  \epsffile{pictures/dialog/procs.ps}}

  \label{fig:The procedure XFProcProgProcs}
\end{figure}
}

\stepcounter{subsection}
{\newpage
\clearpage
\samepage \begin{figure}[hbt]
  \centerline{
  \epsfysize=9cm
  \epsffile{pictures/dialog/showScript.ps}}

  \label{fig:The procedure XFProcProgShowScript}
\end{figure}
}

\stepcounter{subsection}
\stepcounter{subsection}
{\newpage
\clearpage
\samepage \begin{figure}[hbt]
  \centerline{
  \epsfysize=10.5cm
  \epsffile{pictures/dialog/widgetTree.ps}}

  \label{fig:The procedure XFProcProgWidgetTree}
\end{figure}
}

\stepcounter{section}
\stepcounter{subsection}
{\newpage
\clearpage
\samepage \begin{figure}[hbt]
  \centerline{
  \epsfysize=10cm
  \epsffile{pictures/dialog/alias.ps}}

  \label{fig:The procedure XFProcMiscAliases}
\end{figure}
}

\stepcounter{subsection}
\stepcounter{subsection}
\stepcounter{subsection}
\stepcounter{subsection}
{\newpage
\clearpage
\samepage \begin{figure}[hbt]
  \centerline{
  \epsfysize=7.5cm
  \epsffile{pictures/dialog/modules.ps}}

  \label{fig:The procedure XFProcMiscModules}
\end{figure}
}

\stepcounter{subsection}
{\newpage
\clearpage
\samepage \begin{figure}[hbt]
  \centerline{
  \epsfysize=7cm
  \epsffile{pictures/dialog/pixmaps.ps}}

  \label{fig:The procedure XFProcMiscPixmaps}
\end{figure}
}

\stepcounter{subsection}
\stepcounter{subsection}
\stepcounter{section}
\stepcounter{subsection}
{\newpage
\clearpage
\samepage \begin{figure}[hbt]
  \centerline{
  \epsfysize=7cm
  \epsffile{pictures/dialog/optBind.ps}}

  \label{fig:The procedure XFProcOptionsBindings}
\end{figure}
}

{\newpage
\clearpage
\samepage \begin{supertabular}{|l|p{6.8cm}|}
Call configuration     & This is the binding to activate
                         parameter setting. This event works
                         for each widget in the application,
                         and also for some parameter setting
                         fields in the {\em XF }
 parameter
                         dialogs.\\  \hline
Select current widget  & This event is used to make one
                         widget in the application the
                         current widget.\\  \hline
Primary select         & This is the primary (usually used)
                         select event.\\  \hline
Secondary select       & This is the alternative for the
                         primary select. This is only used
                         when the primary select is already
                         used (almost never required).\\  \hline
Third select           & This is the alternative for the
                         primary and secondary select.\\  \hline
Show widget name       & This event allows it to display
                         the widget name of the widget under
                         the mouse pointer in a dialog box
                         (the name is also inserted into the
                         cutbuffer, so it can be pasted).\\  \hline
Remove widget name     & This event must correspond to the
                         ``Show widget name'' event. This
                         event removes the dialog box
                         showing the widget name.\\  \hline
Begin widget moving    & This event starts the interactive
                         placing or sizing of a widget. It
                         must correspond to the other
                         moving and sizing events.\\  \hline
Move widget            & This event is the moving event that
                         is used to update the widget
                         position during the moving. It must
                         correspond to the other
                         moving and sizing events.\\  \hline
End widget moving      & This event ends the interactive
                         placing or sizing of a widget. It
                         must correspond to the other
                         moving and sizing events.\\  \hline
Popup menu (mouse nr.) & This is the number of the mouse
                         button that should be used to
                         display a popup menu. Popup menus
                         are available in the widget tree.\\
\end{supertabular}
}

\stepcounter{subsection}
{\newpage
\clearpage
\samepage \begin{figure}[hbt]
  \centerline{
  \epsfysize=13.5cm
  \epsffile{pictures/dialog/optGeneral.ps}}

  \label{fig:The procedure XFProcOptionsGeneral}
\end{figure}
}

{\newpage
\clearpage
\samepage \begin{supertabular}{|l|p{6.5cm}|}
Auto save                    & The interval slider specifies
                               the interval between two auto
                               saves. The file number slider
                               specifies the number of
                               backup files to be created.
                               The backup files are created
                               in the temporary directory,
                               and they start with an ``as''.\\  \hline
Ask for widget name...       & This checkbutton activates a
                               dialog box, where the user
                               can enter a widget name
                               before a widget is inserted.\\  \hline
Mac like bindings (partially)& This checkbutton activates
                               bindings for entry and text
                               widgets which emulate certain
                               Mac functionalities (Meta-c,
                               Meta-x and Meta-v).\\  \hline
Default geometry manager     & Depending on these buttons,
                               new widgets are inserted
                               using the packer or the
                               placer.\\  \hline
Allow layouting without...   & If this checkbutton is true,
                               layouting of the widgets is
                               only allowed when the layout
                               dialog box is popped up (to
                               prevent erroneous geometry
                               changes). Otherwise the
                               layouting is always
                               possible.\\  \hline
Default geometry manager     & Depending on these buttons,
                               new widgets are displayed,
                               using the packer or the
                               placer.\\  \hline
Layout border width          & With this slider, the sizing
                               border of widgets can be
                               specified. This border is
                               used to size widgets, while
                               the remaining inner area is
                               used to move the widget.\\  \hline
GridX/GridY                  & With these sliders, a grid
                               can be defined for widgets
                               that are layouted with the
                               placer.\\  \hline
Scrollbar side               & Depending on these buttons,
                               scrollbars are displayed left
                               from the controlled widgets
                               or right.\\  \hline
Save options on exit         & If this checkbutton is true,
                               the {\em XF }
 options are saved
                               when the program is stopped.\\  \hline
Save positions on exit       & If this checkbutton is true,
                               the {\em XF }
 window positions are
                               saved when the program is
                               stopped.\\  \hline
Binding show levels          & These checkbuttons specify
                               which levels of bindings are
                               displayed in the binding
                               dialog. The level of a
                               binding is specified by the
                               string ``\# xf ignore me
                               $<$level$>$'' at the
                               beginning of the {\em Tcl/Tk }

                               command.\\  \hline
Procedure show levels        & These checkbuttons specify
                               which levels of procedures
                               are displayed in the
                               procedure dialogs. The level
                               of a procedure is specified
                               by the string ``\# xf ignore
                               me $<$level$>$'' at the
                               beginning of the {\em Tcl/Tk }

                               command.\\  \hline
Bitmap editor                & This entry contains the
                               command that is invoked to
                               start an external bitmap
                               editor. The editor command
                               must contain the string
                               \$xfFileName at the position
                               where the filename which is
                               to be edited should be
                               substituted.\\  \hline
Pixmap editor                & This entry contains the
                               command that is invoked to
                               start an external pixmap
                               editor. The editor command
                               must contain the string
                               \$xfFileName at the position
                               where the filename which is
                               to be edited should be
                               substituted.\\  \hline
Editor                       & This entry contains the
                               command that is invoked to
                               start an external editor. The
                               editor command must contain
                               the string \$xfFileName at
                               the position where the
                               filename which is to be
                               edited should be
                               substituted.\\  \hline
Message font                 & This font is used in {\em XF }

                               message boxes. All other
                               dialogs are using the default
                               font.\\  \hline
Flash color                  & This color is used to
                               highlight the selected
                               widget.\\
\end{supertabular}
}

\stepcounter{subsection}
\stepcounter{subsection}
{\newpage
\clearpage
\samepage \begin{figure}[hbt]
  \centerline{
  \epsfysize=4.7cm
  \epsffile{pictures/dialog/optInterp.ps}}

  \label{fig:The procedure XFProcOptionsInterpreter}
\end{figure}
}

{\newpage
\clearpage
\samepage \begin{supertabular}{|l|p{6.8cm}|}
Motif look \& feel       & This checkbutton toggles the
                          global variable tk\_strict\-Motif
                          which is used to make the behavior
                          of {\em Tk }
 more motif-like.\\  \hline
Interpreter has tkEmacs & Only when this checkbutton is
                          selected, the tkEmacs widget is
                          used for editing {\em Tcl/Tk }
 source.
                          Otherwise, an existing tkEmacs
                          widget is ignored.\\  \hline
Tk handles Kanji Fonts  & This checkbutton enables the kanji
                          font support.\\  \hline
Interpreter             & This is the name of the
                          interpreter which is inserted at
                          the beginning of the created code
                          to allow the execution of this
                          {\em Tcl/Tk }
 code.\\  \hline
Interpreter (editor)    & This is the name of the
                          interpreter that is used to run
                          external editors. Usually, this is
                          the standard wish. To allow the
                          use of special extensions in future
                          versions, this name is adaptable.\\  \hline
Interpreter (testing)   & This is the name of the
                          interpreter that is used to test
                          the scripts. Usually, this is the
                          standard wish. To allow the use of
                          special extensions in future
                          versions, this name is adaptable.\\  \hline
Interpreter (tutorial)  & This is the name of the
                          interpreter that is used to run
                          the tutorial. Usually, this is the
                          standard wish. To allow the use of
                          special extensions in future
                          versions, this name is adaptable.\\  \hline
\end{supertabular}
}

\stepcounter{subsection}
\stepcounter{subsection}
{\newpage
\clearpage
\samepage \begin{figure}[hbt]
  \centerline{
  \epsfysize=12.5cm
  \epsffile{pictures/dialog/optPath.ps}}

  \label{fig:The procedure XFProcOptionsPathFile}
\end{figure}
}

{\newpage
\clearpage
\samepage \begin{supertabular}{|l|p{7.8cm}|}
XF path           & This pathname is pointing at the root of
                    the installed {\em XF }
 distribution.\\  \hline
Additionals path  & This pathname is pointing at the
                    directory where the sources for the
                    support of additional widgets are
                    located.\\  \hline
Elements path     & This pathname is pointing at the
                    directory where the sources for the
                    support of the standard {\em Tk }
 widgets are
                    located.\\  \hline
Help path         & This is a list of pathnames separated by
                    ``:'' containing the help pages for
                    the help program.\\  \hline
Icon path         & This is a list of pathnames separated by
                    ``:'' containing the icons for the
                    iconbar.\\  \hline
Library path      & This pathname points at the directory
                    where the library files of {\em XF }
 are
                    located.\\  \hline
Module load path  & This is a list of pathnames separated by
                    ``:'' pointing at directories where
                    {\em XF }
 can find modules that should be
                    loaded. If these directories contain
                    tclIndex files, the auto loading
                    facility of {\em Tcl }
 also uses this
                    pathname.\\  \hline
Procedures path   & This pathname is pointing at the
                    directory where the {\em Tcl/Tk }
 procedures
                    can be stored.\\  \hline
Source path       & This pathname is pointing at the
                    directory where the {\em XF }
 sources are
                    located.\\  \hline
Template path     & This is a list of pathnames separated by
                    ``:'' pointing at directories where
                    templates can be found and stored.\\  \hline
Tmp path          & This pathname is pointing at the
                    directory where {\em XF }
 can store temporary
                    data. This includes the auto save
                    files.\\  \hline
AppDef file       & This filename specifies the application
                    default file that {\em XF }
 should load at
                    startup. This file can contain standard
                    X resource specifications\\  \hline
Binding file      & This filename specifies the file
                    containing class bindings. These
                    bindings can be changed and saved with
                    {\em XF}
. If the class bindings are
                    significant for the application, they
                    should be included directly in the
                    application source with an option in
                    ({\tt Options $|$ Source options\tt}).\\  \hline
Color file        & This filename specifies the file
                    that contains the color\-names for the
                    color selection box. This file is
                    created automatically when {\em XF }
 is
                    installed.\\  \hline
Config file       & This filename specifies the
                    configuration file for {\em XF}
. This
                    filename can be specified with a
                    commandline option when {\em XF }
 is
                    started (-xfconfig).\\  \hline
Cursor file       & This filename specifies the file
                    that contains the cursor\-names for the
                    cursor selection box. This file is
                    created automatically when {\em XF }
 is
                    installed.\\  \hline
Font file         & This filename specifies the file
                    that contains the font\-names for the
                    font selection box. This file is
                    created automatically when {\em XF }
 is
                    installed.\\  \hline
Iconbar file      & This filename specifies the
                    iconbar configuration file.\\  \hline
Keysym file       & This filename specifies the file
                    that contais the keysym\-names for the
                    keysym selection box. This file is
                    created automatically when {\em XF }
 is
                    installed.\\  \hline
Menubar file      & This filename specifies the
                    menubar configuration file.\\  \hline
Position file     & This filename specifies the
                    window position file for {\em XF}
. This file
                    contains the window positions of the {\em XF }

                    dialog boxes.\\  \hline
Startup file      & This filename specifies the
                    startup file. This file is evaluated
                    when {\em XF }
 is started. Here, the user can
                    make local extensions to {\em XF}
.\\  \hline
TkEmacs editor    & This is the name of the emacs that is
                    called by the tkEmacs widget. Usually,
                    this value is not changed.\\  \hline
TkEmacs lisp file & This is the name of the emacs lisp code
                    that is loaded by the tkEmacs widget.
                    Usually, this value is not changed.\\
\end{supertabular}
}

\stepcounter{subsection}
\stepcounter{subsection}
\stepcounter{subsection}
\stepcounter{subsection}
\stepcounter{subsection}
{\newpage
\clearpage
\samepage \begin{figure}[hbt]
  \centerline{
  \epsfysize=12.5cm
  \epsffile{pictures/dialog/optSource.ps}}

  \label{fig:The procedure XFProcOptionsSource}
\end{figure}
}

{\newpage
\clearpage
\samepage \begin{supertabular}{|l|p{6.2cm}|}
Application default code     & If this checkbutton is
                               selected, {\em XF }
 will create
                               code that allows the parsing
                               of application default files.
                               The code searches in the
                               application default
                               directories for a file
                               matching the application
                               name, and parses it.\\  \hline
Form support code            & If this checkbutton is true,
                               {\em XF }
 will create code that
                               supports formulars. The code
                               allows the automatic
                               connection of text/entry
                               widgets, and handles the
                               geometry of these widgets.\\  \hline
Commandline parsing code     & If this checkbutton is true,
                               {\em XF }
 will create code that
                               parses the commandline
                               options for some special {\em XF }

                               extensions.\\  \hline
Pixmap preloading code       & If this checkbutton is true,
                               {\em XF }
 will create code that
                               uses the TkPixmap extension
                               {\em pinfo\em} to include the
                               bitmaps/pixmaps that are used
                               by the application into the
                               code.\\  \hline
Class bindings               & If this checkbutton is true,
                               {\em XF }
 will include the class
                               bindings into the created
                               code.\\  \hline
Create tclIndex file         & If this checkbutton is true,
                               {\em XF }
 will create a tclIndex
                               file for those modules that
                               are specified to be auto
                               loadable.\\  \hline
Create shell script          & If this checkbutton is true,
                               {\em XF }
 will create a shell
                               script for calling the
                               resulting application.\\  \hline
Bindings are surrounded...   & Depending on these buttons,
                               {\em XF }
 will enclose the {\em Tcl/Tk }

                               commands bound to an event in
                               \{\} or ''''. Please use
                               \{\}, for the enclosing in
                               '''' may lead into trouble.\\  \hline
Procedures are surrounded... & Depending on these buttons,
                               {\em XF }
 will enclose the {\em Tcl/Tk }

                               commands bound to a resource
                               (like the -command resource
                               for buttons) in \{\} or ''''.
                               Please use \{\}, for the
                               enclosing in '''' may lead
                               into trouble.\\  \hline
Binding save levels          & These checkbuttons specify
                               which levels of bindings are
                               saved. The level of a
                               binding is specified by the
                               string ``\# xf ignore me
                               $<$level$>$'' at the
                               beginning of the {\em Tcl/Tk }

                               command.\\  \hline
Procedure save levels        & These checkbuttons specify
                               which levels of the
                               procedures are saved. The
                               level of a procedure is
                               specified by the string ``\#
                               xf ignore me $<$level$>$'' at
                               the beginning of the {\em Tcl/Tk }

                               command.\\  \hline
Comment layout               & Depending on the radiobuttons
                               below the text widget, the
                               text widget allows the
                               adaption of the comments that
                               are inserted in the code by
                               {\em XF}
. These comments can
                               contain several variables.
                               These are: programName,
                               moduleName, tclVersion,
                               tkVersion, xfVersion,
                               magicCookie and procedureName.\\
\end{supertabular}
}

\stepcounter{subsection}
{\newpage
\clearpage
\samepage \begin{figure}[hbt]
  \centerline{
  \epsfysize=7cm
  \epsffile{pictures/dialog/optVersion.ps}}

  \label{fig:The procedure XFProcOptionsVersion}
\end{figure}
}

{\newpage
\clearpage
\samepage \begin{supertabular}{|l|p{7.5cm}|}
Use version control     & This checkbutton allows it to
                          disable the use of the version
                          control system.\\  \hline
List                    & This command is executed to get a
                          name list of all objects in the
                          version system. Before this
                          command is executed, {\em XF }
 changes
                          into the correct directory.\\  \hline
List (long)             & This command is executed to get a
                          detailed information on one
                          specific object in the version
                          system. The object is identified
                          with a version number. Before this
                          command is executed, {\em XF }
 changes
                          into the correct directory.\\  \hline
List default (long)     & This command is executed to get a
                          detailed information on one
                          specific object in the version
                          system. The object is the default
                          object that is used when no
                          explicit version number is given.
                          Before this command is executed,
                          {\em XF }
 changes into the correct
                          directory.\\  \hline
Retrieve                & This command is executed to
                          retrieve one specific object from
                          the version system. The object is
                          identified with a version number.
                          Before this command is executed,
                          {\em XF }
 changes into the correct
                          directory.\\  \hline
Retrieve default        & This command is executed to
                          retrieve one specific object from
                          the version system. The object is
                          the default object that is used
                          when no explicit version number is
                          given. Before this command is
                          executed, {\em XF }
 changes into the
                          correct directory.\\  \hline
Remove                  & This command is executed to
                          remove a retrieved object. Before
                          this command is executed, {\em XF }

                          changes into the correct
                          directory.\\  \hline
Save                    & This command is executed to
                          save an object into the version
                          system. Before this command is
                          executed, {\em XF }
 changes into the
                          correct directory.\\  \hline
Save with comment       & This command is executed to
                          save an object into the version
                          system. It also takes a message
                          that is attached to that object.
                          Before this command is executed,
                          {\em XF }
 changes into the correct
                          directory.\\  \hline
Show                    & This command is executed to
                          show the contents of one specific
                          object from the version system.
                          The object is identified with a
                          version number. Before this
                          command is executed, {\em XF }
 changes
                          into the correct directory.\\  \hline
Show default            & This command is executed to
                          show the contents of one specific
                          object from the version system.
                          The object is the default object
                          that is used when no explicit
                          version number is given. Before
                          this command is executed, {\em XF }

                          changes into the correct
                          directory.\\  \hline
Test                    & This command is executed to
                          check if the version control
                          system is installed on the
                          machine.\\
\end{supertabular}
}

\stepcounter{subsection}
{\newpage
\clearpage
\samepage \begin{figure}[hbt]
  \centerline{
  \epsfysize=9cm
  \epsffile{pictures/dialog/optWindow.ps}}

  \label{fig:The procedure XFProcOptionsWindow}
\end{figure}
}

{\newpage
\clearpage
\samepage \begin{supertabular}{|l|p{5.5cm}|}
Automatic window placing      & This checkbutton toggles the
                                placing policy of {\em XF}
.
                                Automatic placing means,
                                that the position of the
                                dialog boxes is set by {\em XF }

                                at startup. The changes that
                                the user makes are stored.\\  \hline
Automatic window sizing       & This checkbutton toggles the
                                sizing policy of {\em XF}
.
                                Automatic sizing means,
                                that the size of the dialog
                                boxes is set by {\em XF }
 at
                                startup. The changes that
                                the user makes are stored.\\  \hline
Automatic window stacking     & This checkbutton toggles the
                                placing/sizing policy of
                                {\em XF}
. Automatic stacking means
                                that the size and position
                                of some dialog boxes are set
                                to the size and position of
                                a ``leading'' window. This
                                can only be done for
                                parameter dialogs.\\  \hline
One window per window class   & This checkbutton toggles the
                                dialog box creation policy
                                of {\em XF}
. If only one window
                                per window class is allowed,
                                {\em XF }
 will use an already
                                existing toplevel of the
                                same window class to display
                                dialog boxes.\\  \hline
Automatic root window placing & If this checkbutton is true,
                                the main application window
                                is placed to the position
                                +0+0 on startup.\\  \hline
Hide edit lists               & If this checkbutton is true,
                                the main {\em XF }
 window does not
                                contain the widget listboxes.\\  \hline
Hide iconbar                  & If this checkbutton is true,
                                the main {\em XF }
 window does not
                                contain the icon\-bar.\\  \hline
Hide menubar                  & If this checkbutton is true,
                                the main {\em XF }
 window does not
                                contain the menu\-bar.\\  \hline
Hide path name                & If this checkbutton is true,
                                the main {\em XF }
 window does not
                                contain the current widget
                                path.\\  \hline
Hide status line              & If this checkbutton is true,
                                the main {\em XF }
 window does not
                                contain the status line.\\  \hline
Show iconbar as toplevel      & If this checkbutton is true,
                                the icon\-bar of the
                                main {\em XF }
 window is displayed
                                as a separate toplevel at the
                                {\em XF }
 startup.\\
\end{supertabular}
}

\stepcounter{section}
\stepcounter{subsection}
{\newpage
\clearpage
\samepage \begin{figure}[hbt]
  \centerline{
  \epsfysize=5cm
  \epsffile{pictures/dialog/about.ps}}

  \label{fig:The procedure XFProcHelpAbout}
\end{figure}
}

\stepcounter{subsection}
\stepcounter{subsection}
\stepcounter{chapter}
\stepcounter{section}
\stepcounter{subsection}
{\newpage
\clearpage
\samepage \begin{figure}[ht]
  \centerline{
  \epsfysize=10cm
  \epsffile{pictures/templates/CanvasLS.ps}}

  \label{fig:CanvasLS}
\end{figure}
}

\stepcounter{subsection}
{\newpage
\clearpage
\samepage \begin{figure}[ht]
  \centerline{
  \epsfysize=1cm
  \epsffile{pictures/templates/EntryL.ps}}

  \label{fig:EntryL}
\end{figure}
}

{\newpage
\clearpage
\samepage \begin{figure}[ht]
  \centerline{
  \epsfysize=1cm
  \epsffile{pictures/templates/EntryLLS.ps}}

  \label{fig:EntryLLS}
\end{figure}
}

{\newpage
\clearpage
\samepage \begin{figure}[ht]
  \centerline{
  \epsfysize=1cm
  \epsffile{pictures/templates/EntryLS.ps}}

  \label{fig:EntryLS}
\end{figure}
}

{\newpage
\clearpage
\samepage \begin{figure}[ht]
  \centerline{
  \epsfysize=1cm
  \epsffile{pictures/templates/EntryS.ps}}

  \label{fig:EntryS}
\end{figure}
}

\stepcounter{subsection}
{\newpage
\clearpage
\samepage \begin{figure}[ht]
  \centerline{
  \epsfysize=14cm
  \epsffile{pictures/templates/HypertextLS.ps}}

  \label{fig:HypertextLS}
\end{figure}
}

\stepcounter{subsection}
{\newpage
\clearpage
\samepage \begin{figure}[ht]
  \centerline{
  \epsfysize=8cm
  \epsffile{pictures/templates/ListboxLS.ps}}

  \label{fig:ListboxLS}
\end{figure}
}

\stepcounter{subsection}
{\newpage
\clearpage
\samepage \begin{figure}[ht]
  \centerline{
  \epsfysize=7cm
  \epsffile{pictures/templates/PhotoLS.ps}}

  \label{fig:PhotoLS}
\end{figure}
}

\stepcounter{subsection}
{\newpage
\clearpage
\samepage \begin{figure}[ht]
  \centerline{
  \epsfysize=10cm
  \epsffile{pictures/templates/TextLS.ps}}

  \label{fig:TextLS}
\end{figure}
}

\stepcounter{subsection}
{\newpage
\clearpage
\samepage \begin{figure}[ht]
  \centerline{
  \epsfysize=9cm
  \epsffile{pictures/templates/TkEmacsLS.ps}}

  \label{fig:TkEmacsLS}
\end{figure}
}

\stepcounter{section}
\stepcounter{subsection}
{\newpage
\clearpage
\samepage \begin{tabular}{|l|l|p{6.5cm}|} \hline
Parameter name    & Opt. & Purpose\\  \hline
alertBoxMessage   & y    & The message, file or file
                           descriptor that is displayed.\\  \hline
alertBoxCommand	  & y    & The command to execute when OK is
                           pressed. The dialog box is not modal
                           (non blocking) when this parameter
                           is not an empty string.\\  \hline
alertBoxGeometry  & y    & This is the geometry of the dialog
                           box.\\  \hline
alertBoxTitle     & y    & This is the title bar of the
                           dialog box.\\  \hline
args              & y    & Any additional parameters are
                           interpreted as a button label.
                           The dialog box is modal
                           (blocking), and the return value
                           of the procedure is the number of
                           the pressed button.\\  \hline
\end{tabular}
}

{\newpage
\clearpage
\samepage \begin{tabular}{|l|l|p{5.5cm}|} \hline
Array element     & Default   & Purpose\\  \hline
activeBackground  & -         & The active background color.\\  \hline
activeForeground  & -         & The active foreground color.\\  \hline
after             & 0         & Invokes the first button after
                                n seconds. The dialog box
                                is removed.\\  \hline
anchor            & nw        & The anchor of the message widget.\\  \hline
background        & -         & The background color.\\  \hline
font              & -         & The font.\\  \hline
foreground        & -         & The foreground color.\\  \hline
justify           & center    & The justification of the
                                widget displaying the
                                message.\\  \hline
toplevelName      & .alertBox & This variable contains the
                                name of the toplevel widget.
                                It makes it possible to
                                popup multiple dialog boxes
                                at the same time.\\  \hline
\end{tabular}
}

{\newpage
\clearpage
\samepage \setbox\sizebox=\hbox{$\backslash$}\lthtmltypeout{latex2htmlSize :tex2html_wrap_inline2562: \the\ht\sizebox::\the\dp\sizebox.}\box\sizebox
}

{\newpage
\clearpage
\samepage \begin{figure}[ht]
  \centerline{
  \epsfysize=4.2cm
  \epsffile{pictures/templates/AlertBox.ps}}

  \label{fig:AlertBox}
\end{figure}
}

\stepcounter{subsection}
{\newpage
\clearpage
\samepage \begin{tabular}{|l|l|p{6.5cm}|} \hline
Parameter name & Opt. & Purpose \\  \hline
listWidget     & n    & The list/text widget that should be
                        cleared \\  \hline
\end{tabular}
}

\stepcounter{subsection}
{\newpage
\clearpage
\samepage \begin{tabular}{|l|l|p{6.7cm}|} \hline
Parameter name    & Opt. & Purpose\\  \hline
colorBoxFileColor & y    & The file containing a list of
                           colors.\\  \hline
colorBoxMessage   & y    & The message to be displayed. If
                           the parameter contains the
                           patterns *foreground* or
                           *background*, the appropriate
                           resource is set in the demo
                           widget, and in the target
                           widget.\\  \hline
colorBoxEntryW    & y    & This is the entry widget where
                           the selected color is to be
                           inserted.\\  \hline
colorBoxTargetW   & y    & This is the widget that is
                           configured. If this parameter is
                           specified, the selected color is
                           applied to the widget.\\  \hline
\end{tabular}
}

{\newpage
\clearpage
\samepage \begin{tabular}{|l|l|p{5.5cm}|} \hline
Array element          & Default & Purpose\\  \hline
activeBackground       & -       & The active background
                                   color.\\  \hline
activeForeground       & -       & The active foreground
                                   color.\\  \hline
background             & -       & The background color.\\  \hline
font                   & -       & The font.\\  \hline
foreground             & -       & The foreground color.\\  \hline
palette                & ''''    & A list of color names.\\  \hline
scrollActiveForeground & -       & The active foreground
                                   color of the scrollbar.\\  \hline
scrollBackground       & -       & The scrollbar background
                                   color.\\  \hline
scrollForeground       & -       & The scrollbar foreground
                                   color.\\  \hline
scrollSide             & right   & The side of the
                                   scrollbar.\\  \hline
\end{tabular}
}

{\newpage
\clearpage
\samepage \begin{figure}[ht]
  \centerline{
  \epsfysize=6cm
  \epsffile{pictures/templates/ColorBox.ps}}

  \label{fig:ColorBox}
\end{figure}
}

\stepcounter{subsection}
{\newpage
\clearpage
\samepage \begin{tabular}{|l|l|p{6.3cm}|} \hline
Parameter name      & Opt. & Purpose \\  \hline
cursorBoxFileCursor & y    & The file containing a list of
                             cursors. \\  \hline
cursorBoxFileColor  & y    & The file containing a list of
                             colors. \\  \hline
cursorBoxMessage    & y    & The resource name that is
                             configured. \\  \hline
cursorBoxEntryW     & y    & This is the entry widget where
                             the selected cursor is
                             inserted. \\  \hline
cursorBoxTargetW    & y    & This is the widget that is
                             configured. If this parameter
                             is specified, the selected
                             cursor is applied to the widget
                             immediately. \\  \hline
\end{tabular}
}

{\newpage
\clearpage
\samepage \begin{tabular}{|l|l|p{5.5cm}|} \hline
Array element          & Default & Purpose\\  \hline
activeBackground       & -       & The active background
                                   color.\\  \hline
activeForeground       & -       & The active foreground
                                   color.\\  \hline
background             & -       & The background color.\\  \hline
font                   & -       & The font.\\  \hline
foreground             & -       & The foreground color.\\  \hline
scrollActiveForeground & -       & The active foreground
                                   color of the scrollbar.\\  \hline
scrollBackground       & -       & The scrollbar background
                                   color.\\  \hline
scrollForeground       & -       & The scrollbar foreground
                                   color.\\  \hline
scrollSide             & right   & The side of the scrollbar.\\  \hline
\end{tabular}
}

{\newpage
\clearpage
\samepage \begin{figure}[ht]
  \centerline{
  \epsfysize=6.3cm
  \epsffile{pictures/templates/CursorBox.ps}}

  \label{fig:CursorBox}
\end{figure}
}

\stepcounter{subsection}
{\newpage
\clearpage
\samepage \begin{tabular}{|l|l|p{6.5cm}|} \hline
Parameter name    & Opt. & Purpose \\  \hline
fsBoxMessage      & y    & The message to be displayed. \\  \hline
fsBoxFileName     & y    & This is a file name that is
                           inserted in the file name
                           selection field, as a default
                           value \\  \hline
fsBoxActionOk     & y    & This is the Tcl script that is
                           evaluated when the OK button is
                           pressed. To access the selected
                           file and path name, access the
                           global variable fsBox described
                           below. If no commands are
                           specified, the dialog box is
                           modal.\\  \hline
fsBoxActionCancel & y    & This is the Tcl script that is
                           evaluated when the Cancel button
                           is pressed. To access the
                           selected file and path name,
                           access the global variable fsBox
                           described below. If no commands
                           are specified, the dialog box is
                           modal.\\  \hline
\end{tabular}
}

{\newpage
\clearpage
\samepage \begin{supertabular}{|l|l|p{5.5cm}|}
activeBackground       & -       & The active background
                                   color. \\  \hline
activeForeground       & -       & The active foreground
                                   color. \\  \hline
background             & -       & The background color. \\  \hline
font                   & -       & The font. \\  \hline
foreground             & -       & The foreground color. \\  \hline
name                   & ''''    & The name of the selected
                                   file. \\  \hline
path                   & ''''    & The path name of the
                                   selected file. \\  \hline
pattern                & ''''    & The display selection
                                   pattern. \\  \hline
scrollActiveForeground & -       & The active foreground
                                   color of the scrollbar.\\  \hline
scrollBackground       & -       & The scrollbar background
                                   color. \\  \hline
scrollForeground       & -       & The scrollbar foreground
                                   color. \\  \hline
scrollSide             & right   & The side of the
                                   scrollbar. \\  \hline
showPixmaps            & 0       & If this variable is 1,
                                   the selected files are
                                   interpreted as picture
                                   files, and are displayed
                                   in an area right from the
                                   file list. \\  \hline
\end{supertabular}
}

{\newpage
\clearpage
\samepage \begin{figure}[ht]
  \centerline{
  \epsfysize=8cm
  \epsffile{pictures/templates/FSBox.ps}}

  \label{fig:FSBox}
\end{figure}
}

\stepcounter{subsection}
{\newpage
\clearpage
\samepage \begin{tabular}{|l|l|p{6.5cm}|} \hline
Parameter name & Opt. & Purpose \\  \hline
listWidget     & n    & The list widget where the file
                        contents are inserted. \\  \hline
fileInFile     & y    & The filename/filedescriptor that is
                        to be inserted. \\  \hline
\end{tabular}
}

\stepcounter{subsection}
{\newpage
\clearpage
\samepage \begin{tabular}{|l|l|p{6.7cm}|} \hline
Parameter name  & Opt. & Purpose\\  \hline
fontBoxFileFont & y    & The file containing a list of
                         fonts.\\  \hline
fontBoxResource & y    & The resource name that is
                         configured.\\  \hline
fontBoxEntryW   & y    & This is the entry widget where the
                         selected font is inserted.\\  \hline
fontBoxTargetW  & y    & This is the widget that is
                         configured. If this parameters is
                         specified, the selected font is
                         applied to the widget
                         immediately.\\  \hline
\end{tabular}
}

{\newpage
\clearpage
\samepage \begin{tabular}{|l|l|p{5.5cm}|} \hline
Array element          & Default & Purpose\\  \hline
activeBackground       & -       & The active background
                                   color.\\  \hline
activeForeground       & -       & The active foreground
                                   color.\\  \hline
background             & -       & The background color.\\  \hline
font                   & -       & The font.\\  \hline
font-demo              & -       & The demo string.\\  \hline
foreground             & -       & The foreground color.\\  \hline
scrollActiveForeground & -       & The active foreground
                                   color of the scrollbar.\\  \hline
scrollBackground       & -       & The scrollbar background
                                   color.\\  \hline
scrollForeground       & -       & The scrollbar foreground
                                   color.\\  \hline
scrollSide             & right   & The side of the
                                   scrollbar.\\  \hline
\end{tabular}
}

{\newpage
\clearpage
\samepage \begin{figure}[ht]
  \centerline{
  \epsfysize=7cm
  \epsffile{pictures/templates/FontBox.ps}}

  \label{fig:FontBox}
\end{figure}
}

\stepcounter{subsection}
{\newpage
\clearpage
\samepage \begin{tabular}{|l|l|p{6.5cm}|} \hline
Parameter name  & Opt. & Purpose\\  \hline
iconBarUserFile & n    & This file contains the user
                         specific iconbar configuration.
                         This file is written when the user
                         presses the {\tt Save\tt} button.\\  \hline
iconBarFile     & n    & This file contains the
                         fallback iconbar definition. This
                         file is usually global for all
                         users.\\  \hline
iconBarIcons    & n    & A list of path names separated by
                         ``:''. In thse path names, the
                         bitmaps for the icon bar can be
                         found.\\  \hline
\end{tabular}
}

{\newpage
\clearpage
\samepage \begin{tabular}{|l|l|p{6.5cm}|} \hline
Parameter name  & Opt. & Purpose\\  \hline
iconBarName     & n    & The icon bar name. The name
                         identifies a set of icons. An
                         application can contain several
                         iconbars, each under a unique
                         name.\\  \hline
iconBarPath     & y    & The widget path name where the
                         iconbar is located. If the path
                         name is empty, a toplevel is
                         created.\\  \hline
iconBarStatus   & y    & The status of the iconbar. An
                         iconbar can have the status
                         ``child'', which means, that it is
                         inserted to the widget path defined
                         by the previous parameter. The
                         status ``toplevel'' means that the
                         iconbar is displayed in a separate
                         toplevel.\\  \hline
\end{tabular}
}

{\newpage
\clearpage
\samepage \begin{tabular}{|l|l|p{6.5cm}|} \hline
Parameter name  & Opt. & Purpose\\  \hline
iconBarName     & n    & The icon bar name. The name
                         identifies a set of icons. An
                         application can contain several
                         iconbars, each under a unique
                         name.\\  \hline
iconBarPath     & y    & The widget path name where the
                         iconbar should be inserted. If the
                         path name is not empty, the
                         children of this widget are
                         destroyed.\\  \hline
\end{tabular}
}

{\newpage
\clearpage
\samepage \begin{tabular}{|l|l|p{6.5cm}|} \hline
Parameter name  & Opt. & Purpose\\  \hline
iconBarName     & n    & The icon bar name. The name
                         identifies a set of icons. An
                         application can contain several
                         iconbars, each under a unique
                         name.\\  \hline
iconBarPath     & y    & The widget path name where the
                         iconbar is located.\\  \hline
iconBarProcs    & y    & A list of procedure names that can
                         be used in the iconbar. This does
                         not restrict the usage of other
                         procedures, but it gives an
                         impression of the available
                         functionality.\\  \hline
\end{tabular}
}

{\newpage
\clearpage
\samepage \begin{supertabular}{|l|l|p{5.5cm}|}
activeBackground       & -       & The active background
                                   color.\\  \hline
activeForeground       & -       & The active foreground
                                   color.\\  \hline
background             & -       & The background color.\\  \hline
barBorder              & 2       & The iconbar border
                                   width.\\  \hline
barIgnoreSep           & 0       & The iconbar separators
                                   are ignored.\\  \hline
barRelief              & sunken  & The iconbar relief.\\  \hline
font                   & -       & The font.\\  \hline
foreground             & -       & The foreground color.\\  \hline
iconBorder             & 2       & The icon border width.\\  \hline
iconHeight             & 20      & The icon height.\\  \hline
iconOffset             & 0       & The icon offset.\\  \hline
iconRelief             & 2       & The icon relief.\\  \hline
iconWidth              & 20      & The icon width.\\  \hline
label                  & ''''    & The label where the
                                   description is
                                   displayed.\\  \hline
scrollActiveForeground & -       & The active foreground
                                   color of the scrollbar.\\  \hline
scrollBackground       & -       & The scrollbar background
                                   color.\\  \hline
scrollForeground       & -       & The scrollbar foreground
                                   color.\\  \hline
scrollSide             & right   & The side of the
                                   scrollbar.\\  \hline
\end{supertabular}
}

{\newpage
\clearpage
\samepage \begin{figure}[ht]
  \centerline{
  \epsfysize=1.2cm
  \epsffile{pictures/templates/IconBar.ps}}

  \label{fig:IconBar}
\end{figure}
}

{\newpage
\clearpage
\samepage \begin{figure}[ht]
  \centerline{
  \epsfysize=10.5cm
  \epsffile{pictures/templates/IconBarConf.ps}}

  \label{fig:IconBarConf}
\end{figure}
}

\stepcounter{subsection}
{\newpage
\clearpage
\samepage \begin{tabular}{|l|l|p{6.3cm}|} \hline
Parameter name    & Opt. & Purpose\\  \hline
inputBoxMessage   & y    & The message to be displayed.\\  \hline
inputBoxCmdOk     & y    & The Tcl script that is
                           evaluated when the button
                           ({\tt OK\tt}) is pressed.
                           To access the inserted text,
                           the variable
                           input\-Box(top\-level\-Name,
                           input\-One) or
                           input\-Box(top\-level\-Name,
                           input\-Multi) (no spaces)
                           is used. If no commands are
                           specified, the dialog box is
                           modal.\\  \hline
inputBoxCmdCancel & y    & The Tcl script that is
                           evaluated when the button
                           ({\tt Cancel\tt}) is
                           pressed. To access the
                           inserted text, the variable
                           input\-Box(top\-level\-Name,
                           input\-One) or
                           input\-Box(top\-level\-Name,
                           input\-Multi) (no spaces) is
                           used. If no commands are
                           specified, the dialog box is
                           modal.\\  \hline
inputBoxGeometry  & y    & The geometry of the toplevel.\\  \hline
inputBoxTitle     & y    & The title of the toplevel.\\  \hline
\end{tabular}
}

{\newpage
\clearpage
\samepage \begin{supertabular}{|l|l|p{4.7cm}|}
activeBackground        & -         & The active background
                                      color.\\  \hline
activeForeground        & -         & The active foreground
                                      color.\\  \hline
anchor                  & n         & The anchor of the
                                      widget that displays
                                      the message.\\  \hline
background              & -         & The background
                                      color.\\  \hline
font                    & -         & The font.\\  \hline
foreground              & -         & The foreground
                                      color.\\  \hline
scrollActiveForeground  & -         & The active foreground
                                      color of the scrollbar.\\  \hline
scrollBackground        & -         & The background color
                                      of the scrollbar.\\  \hline
scrollForeground        & -         & The foreground color
                                      of the scrollbar.\\  \hline
scrollSide              & right     & The side of the
                                      scrollbar.\\  \hline
toplevelName            & .inputBox & This variable contains
                                      the top\-level widget
                                      name. It makes it
                                      possible to popup
                                      multiple dialog boxes
                                      at the same time.\\  \hline
toplevelName,inputOne   & ''''      & ``toplevelName'' is
                                      replaced by the name
                                      of the toplevel. This
                                      variable contains the
                                      text of the one line
                                      input box.\\  \hline
toplevelName,inputMulti & ''''      & ``toplevelName'' is
                                      replaced by the name
                                      of the toplevel. This
                                      variable contains the
                                      text of the multiple
                                      line input box.\\  \hline
\end{supertabular}
}

{\newpage
\clearpage
\samepage \begin{figure}[ht]
  \centerline{
  \epsfysize=4cm
  \epsffile{pictures/templates/InputBox.ps}}

  \label{fig:InputBox}
\end{figure}
}

\stepcounter{subsection}
{\newpage
\clearpage
\samepage \begin{tabular}{|l|l|p{6.5cm}|} \hline
Parameter name & Opt. & Purpose \\  \hline
pathName       & n    & The path/file name to check. \\
                        \hline
\end{tabular}
}

\stepcounter{subsection}
{\newpage
\clearpage
\samepage \begin{tabular}{|l|l|p{6cm}|} \hline
Parameter name      & Opt. & Purpose\\  \hline
keysymBoxFileKeysym & y    & The file containing a list of
                             keysyms.\\  \hline
keysymBoxMessage    & y    & The message to display.\\  \hline
keysymBoxEntryW     & y    & The entry widget where the
                             selected keysym is inserted.\\  \hline
\end{tabular}
}

{\newpage
\clearpage
\samepage \begin{tabular}{|l|l|p{5.5cm}|} \hline
Array element          & Default & Purpose\\  \hline
activeBackground       & -       & The active background
                                   color.\\  \hline
activeForeground       & -       & The active foreground
                                   color.\\  \hline
background             & -       & The background color.\\  \hline
font                   & -       & The font. \\  \hline
foreground             & -       & The foreground color.\\  \hline
overwrite              & 0       & New events are inserted
                                   into the entry widget, or
                                   overwrite the current
                                   event.\\  \hline
scrollActiveForeground & -       & The active foreground
                                   color of the scrollbar.\\  \hline
scrollBackground       & -       & The scrollbar background
                                   color.\\  \hline
scrollForeground       & -       & The scrollbar foreground
                                   color.\\  \hline
scrollSide             & right   & The side of the
                                   scrollbar.\\  \hline
\end{tabular}
}

{\newpage
\clearpage
\samepage \begin{figure}[ht]
  \centerline{
  \epsfysize=7.5cm
  \epsffile{pictures/templates/KeysymBox.ps}}

  \label{fig:KeysymBox}
\end{figure}
}

\stepcounter{subsection}
{\newpage
\clearpage
\samepage \begin{tabular}{|l|l|p{6.5cm}|} \hline
Parameter name & Opt. & Purpose \\  \hline
widgetName     & n    & The name of the menubutton that is
                        to be created.\\  \hline
buttonLabel    & n    & The label of the menubutton.\\  \hline
itemType       & n    & The type of the menu items that are
                        created. Valid types are command,
                        check and radio.\\  \hline
itemList       & n    & The list of menu item names that are
                        to be created. If itemType is check
                        or radio, and the itemFunctions are
                        empty, these are also the names of
                        the associated variable.\\  \hline
itemFunctions  & y    & This list contains one or more
                        command or variable names. They are
                        attached to the created menu
                        items.\\  \hline
\end{tabular}
}

{\newpage
\clearpage
\samepage \begin{figure}[ht]
  \centerline{
  \epsfysize=4cm
  \epsffile{pictures/templates/MakeMButton.ps}}

  \label{fig:MakeMButton}
\end{figure}
}

\stepcounter{subsection}
{\newpage
\clearpage
\samepage \begin{tabular}{|l|l|p{6.5cm}|} \hline
Parameter name  & Opt. & Purpose\\  \hline
menuBarUserFile & n    & This file contains the
                         user-specific menubar
                         configuration. This file is written
                         when the user presses the
                         {\tt Save\tt} button.\\  \hline
menuBarFile     & n    & This file contains the fallback
                         menubar definition. This file is
                         usually global for all users.\\  \hline
\end{tabular}
}

{\newpage
\clearpage
\samepage \begin{tabular}{|l|l|p{6.5cm}|} \hline
Parameter name & Opt. & Purpose \\  \hline
menuBarConfig  & n    & The widget path name, containing the
                        menubuttons to be configured.\\  \hline
\end{tabular}
}

{\newpage
\clearpage
\samepage \begin{tabular}{|l|l|p{5.5cm}|} \hline
Array element          & Default & Purpose\\  \hline
activeBackground       & -       & The active background
                                   color.\\  \hline
activeForeground       & -       & The active foreground
                                   color.\\  \hline
background             & -       & The background color.\\  \hline
font                   & -       & The font.\\  \hline
foreground             & -       & The foreground color.\\  \hline
scrollActiveForeground & -       & The active foreground
                                   color of the scrollbar.\\  \hline
scrollBackground       & -       & The scrollbar background
                                   color.\\  \hline
scrollForeground       & -       & The scrollbar foreground
                                   color.\\  \hline
scrollSide             & right   & The side of the
                                   scrollbar.\\  \hline
\end{tabular}
}

{\newpage
\clearpage
\samepage \setbox\sizebox=\hbox{$\sim$}\lthtmltypeout{latex2htmlSize :tex2html_wrap_inline2564: \the\ht\sizebox::\the\dp\sizebox.}\box\sizebox
}

{\newpage
\clearpage
\samepage \begin{figure}[ht]
  \centerline{
  \epsfysize=.8cm
  \epsffile{pictures/templates/MenuBar.ps}}

  \label{fig:MenuBar}
\end{figure}
}

{\newpage
\clearpage
\samepage \begin{figure}[ht]
  \centerline{
  \epsfysize=11.5cm
  \epsffile{pictures/templates/MenuBarConf.ps}}

  \label{fig:MenuBarConf}
\end{figure}
}

\stepcounter{subsection}
{\newpage
\clearpage
\samepage \begin{tabular}{|l|l|p{5.5cm}|} \hline
Array element          & Default & Purpose \\  \hline
activeBackground       & -       & The active background
                                   color. \\  \hline
activeForeground       & -       & The active foreground
                                   color. \\  \hline
background             & -       & The background color. \\  \hline
font                   & -       & The font. \\  \hline
foreground             & -       & The foreground color. \\  \hline
scrollActiveForeground & -       & The active foreground
                                   color of the scrollbar.\\  \hline
scrollBackground       & -       & The scrollbar background
                                   color. \\  \hline
scrollForeground       & -       & The scrollbar foreground
                                   color. \\  \hline
scrollSide             & right   & The side of the
                                   scrollbar. \\  \hline
\end{tabular}
}

{\newpage
\clearpage
\samepage \begin{figure}[ht]
  \centerline{
  \epsfysize=4cm
  \epsffile{pictures/templates/ReadBox.ps}}

  \label{fig:ReadBox}
\end{figure}
}

\stepcounter{subsection}
{\newpage
\clearpage
\samepage \begin{tabular}{|l|l|p{6.5cm}|} \hline
Parameter name  & Opt. & Purpose\\  \hline
textBoxMessage  & y    & The message, file or file
                         descriptor that is displayed.\\  \hline
textBoxCommand  & y    & The command to be executed when OK
                         is pressed. The dialog box is not
                         modal (non blocking) when this
                         parameter is not an empty
                         string.\\  \hline
textBoxGeometry & y    & This is the geometry of the dialog
                         box.\\  \hline
textBoxTitle    & y    & This is the title bar of the dialog
                         box.\\  \hline
args            & y    & Any additional parameters are
                         interpreted as a button label. The
                         dialog box is modal (blocking).
                         The return value of TextBox
                         is the number of the pressed
                         button.\\  \hline
\end{tabular}
}

{\newpage
\clearpage
\samepage \begin{supertabular}{|l|l|p{5.3cm}|}
activeBackground       & -        & The active background
                                    color.\\  \hline
activeForeground       & -        & The active foreground
                                    color.\\  \hline
background             & -        & The background color.\\  \hline
font                   & -        & The font.\\  \hline
foreground             & -        & The foreground color.\\  \hline
scrollActiveForeground & -        & The active foreground
                                    color of the scrollbar.\\  \hline
scrollBackground       & -        & The scrollbar background
                                    color.\\  \hline
scrollForeground       & -        & The scrollbar foreground
                                    color.\\  \hline
scrollSide             & right    & The side of the
                                    scrollbar.\\  \hline
state                  & disabled & The state of the text
                                    widget. Disabled means,
                                    that no input from the
                                    user is allowed. Normal
                                    means that the user can
                                    type text.\\  \hline
toplevelName           & .textBox & The toplevel name. This
                                    variable makes it
                                    possible to popup
                                    several dialog boxes at
                                    the same time.\\  \hline
\end{supertabular}
}

{\newpage
\clearpage
\samepage \begin{figure}[ht]
  \centerline{
  \epsfysize=3.3cm
  \epsffile{pictures/templates/TextBox.ps}}

  \label{fig:TextBox}
\end{figure}
}

\stepcounter{subsection}
{\newpage
\clearpage
\samepage \begin{tabular}{|l|l|p{6.5cm}|} \hline
Parameter name   & Opt. & Purpose\\  \hline
yesNoBoxMessage  & y    & The message to be displayed.\\  \hline
yesNoBoxGeometry & y    & The geometry of the yes/no box.\\  \hline
\end{tabular}
}

{\newpage
\clearpage
\samepage \begin{tabular}{|l|l|p{5.5cm}|} \hline
Array element    & Default & Purpose \\  \hline
activeBackground & -       & The active background
                             color.\\  \hline
activeForeground & -       & The active foreground
                             color.\\  \hline
afterNo          & 0       & Invokes the no button after
                             n seconds. The dialog box
                             is removed.\\  \hline
afterYes         & 0       & Invokes the yes button after
                             n seconds. The dialog box
                             is removed.\\  \hline
anchor           & nw      & The anchor of the message
                             widget.\\  \hline
background       & -       & The background color.\\  \hline
font             & -       & The font.\\  \hline
foreground       & -       & The foreground color.\\  \hline
justify          & center  & The justification of the
                             widget displaying the
                             message.\\  \hline
\end{tabular}
}

{\newpage
\clearpage
\samepage \begin{figure}[ht]
  \centerline{
  \epsfysize=4cm
  \epsffile{pictures/templates/YesNoBox.ps}}

  \label{fig:YesNoBox}
\end{figure}
}

\stepcounter{subsection}
{\newpage
\clearpage
\samepage \begin{tabular}{|l|l|p{6.5cm}|} \hline
Parameter name  & Opt. & Purpose \\  \hline
cmd             & y    & This command is evaluated when
                         the OK button is pressed\\  \hline
purpose         & y    & This is the message of the
                         file selector box\\  \hline
w               & y    & This is the toplevel path
                         name\\  \hline
\end{tabular}
}

{\newpage
\clearpage
\samepage \begin{figure}[ht]
  \centerline{
  \epsfysize=6.5cm
  \epsffile{pictures/templates/fileselect.ps}}

  \label{fig:fileselect}
\end{figure}
}

\stepcounter{section}
\stepcounter{subsection}
{\newpage
\clearpage
\samepage \setbox\sizebox=\hbox{$Motif^{TM}$}\lthtmltypeout{latex2htmlSize :tex2html_wrap_inline2566: \the\ht\sizebox::\the\dp\sizebox.}\box\sizebox
}

{\newpage
\clearpage
\samepage \begin{figure}[ht]
  \centerline{
  \epsfysize=5cm
  \epsffile{pictures/templates/MListbox.ps}}

  \label{fig:MListbox}
\end{figure}
}

\stepcounter{subsection}
{\newpage
\clearpage
\samepage \setbox\sizebox=\hbox{$Motif^{TM}$}\lthtmltypeout{latex2htmlSize :tex2html_wrap_inline2568: \the\ht\sizebox::\the\dp\sizebox.}\box\sizebox
}

{\newpage
\clearpage
\samepage \begin{figure}[ht]
  \centerline{
  \epsfysize=5cm
  \epsffile{pictures/templates/MListbox2.ps}}

  \label{fig:MListbox2}
\end{figure}
}

\stepcounter{subsection}
{\newpage
\clearpage
\samepage \begin{figure}[ht]
  \centerline{
  \epsfysize=.8cm
  \epsffile{pictures/templates/Menubar.ps}}

  \label{fig:Menubar}
\end{figure}
}

\stepcounter{subsection}
{\newpage
\clearpage
\samepage \begin{figure}[ht]
  \centerline{
  \epsfysize=.9cm
  \epsffile{pictures/templates/OptionButtonE.ps}}

  \label{fig:OptionButtonE}
\end{figure}
}

\stepcounter{subsection}
{\newpage
\clearpage
\samepage \begin{figure}[ht]
  \centerline{
  \epsfysize=3.5cm
  \epsffile{pictures/templates/Popup1.ps}}

  \label{fig:Popup1}
\end{figure}
}

\stepcounter{chapter}

\end{document}
